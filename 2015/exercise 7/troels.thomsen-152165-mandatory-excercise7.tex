\documentclass[12pt]{article}
\usepackage[a4paper, hmargin={2.5cm, 2.5cm}, vmargin={2.5cm, 2.5cm}]{geometry}

\usepackage[utf8]{inputenc}
\usepackage[english]{babel}
\usepackage{amssymb}
\usepackage{setspace}
\usepackage{algorithm}
\usepackage[noend]{algpseudocode}

\usepackage{tikz}
\usetikzlibrary{positioning,shapes, shadows, arrows, automata}

\usepackage{xcolor}
\usepackage{listings}
\usepackage{graphicx}
\usepackage[hidelinks]{hyperref}
\usepackage{float}
\usepackage[english]{varioref}
\usepackage{multirow}
\usepackage{hhline}
\usepackage{etoolbox}

\usepackage{fancyhdr}

\setlength\parindent{0pt}
\usepackage[parfill]{parskip}

\definecolor{mygray}{rgb}{0.9451,0.9451,0.9451}
\lstset{
  backgroundcolor=\color{mygray},
  basicstyle=\footnotesize\ttfamily,
  mathescape,
  breaklines=true,
  numbers=left,
  numberstyle=\ttfamily,
  stepnumber=1,
  firstnumber=1,
  numberfirstline=true,
  postbreak=\raisebox{0ex}[0ex][0ex]{\ensuremath{\color{red}\hookrightarrow\space}},
  literate={->}{$\rightarrow$}{2}
           {ε}{$\varepsilon$}{1}
}

\linespread{1.3}

\title{
  \vspace{4cm}
  \begin{flushleft}
  \Large{\textbf{Mandatory Excercise 7}} \\
  \large{Computationally hard problems}
  \end{flushleft}
  \vspace{0cm}
  \begin{flushleft}
  \small
  \textit{\today}
  \end{flushleft}
  \vspace{12cm}
  \begin{flushleft}
  \small
  Troels Thomsen \texttt{152165} \\
  Solution discussed with Rasmus Haarslev \texttt{152175}
  \end{flushleft}
}

\date{
	%
}

\begin{document}

\clearpage
\pagenumbering{gobble}
\thispagestyle{empty}
\maketitle

\newpage

\pagenumbering{arabic}

\section{Excercise 7.4}

In order to show the Jacobi symbol of $\Big[\frac{773}{1373}\Big]$ we follow the rules outlined in the lecture notes Theorem 5.23. We note that we are also given that $gcd(773, 1373) = 1$.

On the left side of the : we have which rule we utilized, and on the right side the calculation and result.

\begin{eqnarray}
    I3 : & (-1)^{\frac{772}{2}\times\frac{1372}{2}} \Big[\frac{1373}{773}\Big] & = \Big[\frac{1373}{773}\Big] \\
    I2 : & 1373 \bmod 773 & = \Big[\frac{600}{773}\Big] \\
    I1 : & \Big[\frac{300}{773}\Big] \Big[\frac{2}{773}\Big] & = \Big[\frac{300}{773}\Big](-1) \\
    I1 : & \Big[\frac{150}{773}\Big](-1) \times \Big[\frac{2}{773}\Big] & = \Big[\frac{150}{773}\Big] \\
    I1 : & \Big[\frac{75}{773}\Big] \times \Big[\frac{2}{773}\Big] & = \Big[\frac{75}{773}\Big](-1) \\
    I3 : & (-1)(-1)^{\frac{74}{2} \times \frac{772}{2}}\Big[\frac{773}{75}\Big] & = \Big[\frac{773}{75}\Big](-1) \\
    I2 : & 773 \bmod 75 & = \Big[\frac{23}{75}\Big](-1) \\
    I3 : & (-1)^{\frac{22}{2} \times \frac{74}{2}}\Big[\frac{75}{23}\Big] & = \Big[\frac{75}{23}\Big] \\
    I2 : & 75 \bmod 23 & = \Big[\frac{6}{23}\Big] \\
    I1 : & \Big[\frac{3}{23}\Big] \times \Big[\frac{2}{23}\Big] & = \Big[\frac{3}{23}\Big] \\
    I3 : & (-1)^{\frac{2}{2} \times \frac{22}{2}}\Big[\frac{23}{3}\Big] & = \Big[\frac{23}{3}\Big](-1) \\
    I2 : & 23 \bmod 3 & = \Big[\frac{2}{3}\Big](-1) \\
    I5 : & 3 \bmod 8 & = 1
\end{eqnarray}

We see that the Jacobi symbol for $\Big[\frac{773}{1373}\Big]$ is $1$.

\end{document}
