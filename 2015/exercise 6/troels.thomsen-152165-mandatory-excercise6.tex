\documentclass[12pt]{article}
\usepackage[a4paper, hmargin={2.5cm, 2.5cm}, vmargin={2.5cm, 2.5cm}]{geometry}

\usepackage[utf8]{inputenc}
\usepackage[english]{babel}
\usepackage{amssymb}
\usepackage{amsfonts}
\usepackage{amsmath}
\usepackage{setspace}
\usepackage{algorithm}
\usepackage[noend]{algpseudocode}

\usepackage{tikz}
\usetikzlibrary{positioning,shapes, shadows, arrows, automata}

\usepackage{xcolor}
\usepackage{listings}
\usepackage{graphicx}
\usepackage[hidelinks]{hyperref}
\usepackage{float}
\usepackage[english]{varioref}
\usepackage{multirow}
\usepackage{hhline}
\usepackage{etoolbox}

\usepackage{fancyhdr}

\setlength\parindent{0pt}
\usepackage[parfill]{parskip}

\definecolor{mygray}{rgb}{0.9451,0.9451,0.9451}
\lstset{
  backgroundcolor=\color{mygray},
  basicstyle=\footnotesize\ttfamily,
  mathescape,
  breaklines=true,
  numbers=left,
  numberstyle=\ttfamily,
  stepnumber=1,
  firstnumber=1,
  numberfirstline=true,
  postbreak=\raisebox{0ex}[0ex][0ex]{\ensuremath{\color{red}\hookrightarrow\space}},
  literate={->}{$\rightarrow$}{2}
           {ε}{$\varepsilon$}{1}
}

\linespread{1.3}

\title{
  \vspace{4cm}
  \begin{flushleft}
  \Large{\textbf{Mandatory Excercise 6}} \\
  \large{Computationally hard problems}
  \end{flushleft}
  \vspace{0cm}
  \begin{flushleft}
  \small
  \textit{\today}
  \end{flushleft}
  \vspace{12cm}
  \begin{flushleft}
  \small
  Troels Thomsen \texttt{152165} \\
  Solution discussed with Rasmus Haarslev \texttt{152175}
  \end{flushleft}
}

\date{
	%
}

\begin{document}

\clearpage
\pagenumbering{gobble}
\thispagestyle{empty}
\maketitle

\newpage

\pagenumbering{arabic}

\section{Excercise 6.5}

\subsubsection{a}
\label{subs:a}

The general form of the probability for $T = 1$ can be described as follows.

\begin{equation}
    p = 1 - \Big(\frac{2^n-1}{2^n}\Big)^S
\end{equation}

In order to find the $S$ where $p \geq \frac{1}{2}$ we simply solve for
\begin{eqnarray}
    \frac{1}{2} & \leq & 1 - \Big(\frac{2^3-1}{2^3}\Big)^S \\
    \frac{1}{2} & \leq & 1 - \Big(\frac{7}{8}\Big)^S \\
    S & \geq & \frac{ln(2)}{3 ln(2) - log(7)} \\
    S & \geq & {\raise.17ex\hbox{$\scriptstyle\mathtt{\sim}$}} 5.19
\end{eqnarray}

Which tells us that we need at least 6 iterations of the $S$-loop for a probability of at least $\frac{1}{2}$ of all clauses being satisfied.

\subsubsection{b}
\label{subs:b}

Starting from $\{x_1 = 0, x_2 = 0, x_3 = 0\}$, we have to reach the satisfying solution $\{x_1 = 1, x_2 = 1, x_3 = 1\}$ having all of the x’s inversed.

To determine a transformation $T$ for reaching the solution with a probability of at least 12, we look at all possible outcomes of the algorithm in each iteration. So, for $T = 1$ we start in ’000’ and end with three possible outcomes: ’100’, ’010’, and ’001’.

If we have $T = 2$, these three outcomes would each grow to three new outcomes and so on. As soon as we reach a solution, we say that it continues growing with $T$ to three new outcomes (by a factor of 3), that are the same solution.

Since we have three x’s the algorithm will find the first solutions at $T = 3$. By choosing $T = 7$, more than half of all outcomes will be a satisfying solution, which means the algorithm will find a satisfying solution with a probability of at least $\frac{1}{2}$

\end{document}
